\documentclass[a4paper,draft=false,oneside,12pt,ngerman]{scrreprt}
\usepackage[dvips]{graphics}
\usepackage{graphicx}
\usepackage[ngerman]{babel}
\usepackage[section]{placeins}
\usepackage[framed,thmmarks]{ntheorem}

\usepackage[style=authoryear,natbib=true,hyperref=true]{biblatex}
\addbibresource{bibliog.bib}

\usepackage[automark,headsepline,plainheadsepline]{scrpage2}
\automark[section]{chapter}
% Header + Footer Layout
\pagestyle{scrheadings}
\clearscrheadfoot{}
\setheadwidth{textwithmarginpar}
% \setkomafont{pagenumber}{\bfseries}
\setkomafont{pageheadfoot}{\normalfont\sffamily}
\ihead[\leftmark]{\leftmark{}}
% \chead[]{\rightmark}
\ohead[]{\pagemark{}}
\setheadsepline[head]{0.4pt}

\setlength{\parindent}{0cm}

\usepackage{makeidx}
\makeindex{}

%\usepackage[ngerman]{babel}
\usepackage[utf8]{inputenc}
\usepackage[T1]{fontenc}
\usepackage{lmodern}

\usepackage{booktabs}
\usepackage{listings}
\usepackage{verbatim}
\usepackage{needspace}
\usepackage{float}
\usepackage{xcolor}
\usepackage{textcomp}

% \usepackage[hyphens]{url}
\usepackage{csquotes}

\usepackage{mathtools}
\usepackage{mathrsfs}
\usepackage{amsmath}
\usepackage{amsfonts}
\usepackage{amssymb}
\usepackage{amstext}
\usepackage{cancel}

\usepackage[binary-units=true]{siunitx}
\usepackage{eurosym}
\usepackage[]{units}

% SI Symbols configuration
\sisetup{%
  output-decimal-marker = {,},
  per-mode = symbol,
  group-separator = {},
  per=frac,
  retain-explicit-plus,
  range-phrase={ bis },
  range-units=single
}

\graphicspath{{./images/}}

\usepackage{hyperref}
\hypersetup{
    % bookmarks=true,         % show bookmarks bar?
    unicode=false,          % non-Latin characters in Acrobat’s bookmarks
    pdftoolbar=true,        % show Acrobat’s toolbar?
    pdfmenubar=true,        % show Acrobat’s menu?
    pdffitwindow=false,     % window fit to page when opened
    pdfstartview={FitH},    % fits the width of the page to the window
    pdfnewwindow=true,      % links in new window
    colorlinks=false,       % false: boxed links; true: colored links
    linkcolor=red,          % color of internal links (change box color with linkbordercolor)
    linktoc=all,
    citecolor=green,        % color of links to bibliography
    filecolor=magenta,      % color of file links
    urlcolor=cyan           % color of external links
}

\usepackage[all]{hypcap}
\usepackage{breakurl}


\usepackage{tikz}
\usetikzlibrary{shapes,arrows,positioning}
\tikzset{%
  block/.style    = {draw, thick, rectangle, minimum height = 3em, minimum width = 3em}
}

\title{Klausur Planen und Entscheiden SS~2009 Lösung}
\author{Jan~Strohbeck \and Michael~Kaps \and Tobias~Häußer \and Dominik~Bergen \and Kowsikan~Sathiyamoorthy}
\date{\today{}, Aalen}

% Uncomment to speed up compiling process
% \includeonly{%
% % Add include files here
% }

\begin{document}

\pagestyle{empty}
\renewcommand*{\chapterpagestyle}{scrheadings}
\maketitle
\cleardoublepage{}
\pagenumbering{Alph}
\phantomsection{}
\addcontentsline{toc}{chapter}{Inhaltsverzeichnis}
\markboth{Inhaltsverzeichnis}{}%
\tableofcontents
\cleardoublepage{}


\cleardoublepage{}
\renewcommand*{\chapterpagestyle}{scrheadings}
\pagestyle{scrheadings}
\pagenumbering{arabic}

% Text here

% Aufgabe
\chapter{Aufgabe}
\label{chp:aufgabe1}

% - Erreicht der Agent immer sein Ziel?
\section{Erreicht der Agent immer sein Ziel?}
\label{sec:erreicht_der_agent_immer_sein_ziel?}

Nein, da er so unter Umständen in eine Sackgasse gelangen könnte und nicht mehr
herausfinden würde (s.\ untenstehende Skizze).

\begin{center}
\vspace{0.4cm}
\begin{tikzpicture}[auto, thick, node distance=4cm, >=triangle 45]
\draw
    node [block] (start) {Start}
	node [block, right of=start] (ca) {Stadt A}
	node [block, right of=ca] (target) {Ziel}
	node [block, below of=start] (cb) {Stadt B}
	node [block, below of=target] (cc) {Stadt C};
	\draw[->](start) -- node {} (ca);
	\draw[->](start) -- node {} (cb);
	\draw[->](cb) -- node {} (cc);
	\draw[->](cc) -- node {} (target);
\end{tikzpicture}
\end{center}

So würde der Agent immer zwischen den Städten \enquote{Start} und \enquote{Stadt
A} pendeln und nie sein Ziel erreichen.

% - Ordnen Sie dem Agenten einen Agententyp zu
\section{Ordnen Sie dem Agenten einen Agententyp zu}
\label{sec:ordnen_sie_dem_agenten_einen_agententyp_zu}

$\rightarrow$ Reflexagent

% - Arbeitsumgebung
\section{Arbeitsumgebung}
\label{sec:arbeitsumgebung}

\begin{description}
    \item[Performance] Der Agent muss auf dem kürzesten Weg das Ziel
        erreichen\\
        $\rightarrow$ z.B.\ zurückgelegte Strecke wird negativ bewertet,
        höchster Wert (kleinste Strecke) ist der höchste Wert
    \item[Environment] Städte, Straßen zwischen Städten
    \item[Actuators] Möglichkeit, sich in eine Stadt zu bewegen, die mit der
        aktuellen Stadt über eine Straße verbunden ist
    \item[Sensors]
        \begin{itemize}
            \item spezieller Kompass, der immer auf das Ziel zeigt
            \item erkennt von der aktuellen Stadt ausgehende Straßen
        \end{itemize}
\end{description}


% - Eigenschaften der Arbeitsumgebung
\section{Eigenschaften der Arbeitsumgebung}
\label{sec:eigenschaften_der_arbeitsumgebung}

\begin{description}
    \item[teilweise beobachtbar]
        Es ist nicht sofort bekannt, welche Städte und Straßen existieren (nur
        die Start-Stadt und davon ausgehende Straßen).
    \item[deterministisch]
        Kein Zufall, Umgebung ändert sich nicht stategisch bzw.\ gar nicht
    \item[eqisodisch]
        Für den Agenten hängen seine Entscheidungen nicht von den vorherigen
        Entscheidungen ab.
    \item[statisch]
        Es ändern sich z.B.\ keine Verbindungen zwischen Städten zur Laufzeit,
        alles ist statisch.
    \item[diskret]
        Der Agent kann sich nur in diskreten Zuständen befinden (in Städten),
        nicht etwa zwischen Städten. Die Zeitintervalle zwischen Aktionen sind
        auch diskret.
    \item[Einzelagent]
        Es existieren keine weiteren Agenten.
\end{description}

% - Optimale Landkarte
\section{Optimale Landkarte}
\label{sec:optimale_landkarte}

\begin{center}
\vspace{0.4cm}
\begin{tikzpicture}[auto, thick, node distance=4cm, >=triangle 45]
\draw
    node [block] (start) {Start}
	node [block, right of=start] (ca) {Stadt A}
	node [block, right of=ca] (target) {Ziel};
	\draw[->](start) -- node {} (ca);
	\draw[->](ca) -- node {} (target);
\end{tikzpicture}
\end{center}

Agent bewegt sich über Stadt A direkt zum Ziel. Dies ist die optimale Lösung.

% - Nicht optimale Landkarte
\section{Nicht optimale Landkarte}
\label{sec:nicht_optimale_landkarte}

\begin{center}
\vspace{0.4cm}
\begin{tikzpicture}[auto, thick, node distance=4cm, >=triangle 45]
\draw
    node [block] (start) {Start}
	node [block, below of=start] (ca) {Stadt A}
	node [block, below left=of ca] (cb) {Stadt B}
	node [block, below right=of start] (cc) {Stadt C}
	node [block, below right=of cb] (target) {Ziel}
    ;
	\draw[->](start) -- node {1} (ca);
    \draw[->](ca) -- node {$\sqrt{2}$} (cb);
    \draw[->](cb) -- node {$\sqrt{2}$} (target);
    \draw[->](start) -- node {$\sqrt{2}$} (cc);
    \draw[->](cc) -- node {$\sqrt{3}$} (target);
\end{tikzpicture}
\end{center}

Weg des Agenten (Start $\rightarrow$ Stadt A $\rightarrow$ Stadt B $\rightarrow$ Ziel):
\[ w_1 = 1 + 2 \cdot \sqrt{2} \approx 3.828427125 \]

Optimaler Weg (Start $\rightarrow$ Stadt C):
\[ w_o = \sqrt{2} + \sqrt{3} \approx 3.14626437 \]

Die Karte aus Kap.~\ref{sec:vs._zufalls-agent} passt hier auch, da der Agent nie
ins Ziel kommt, was ebenfalls nicht optimal ist.

% - vs. Zufalls-Agent
\section{vs.\ Zufalls-Agent}
\label{sec:vs._zufalls-agent}

\begin{center}
\vspace{0.4cm}
\begin{tikzpicture}[auto, thick, node distance=4cm, >=triangle 45]
\draw
    node [block] (start) {Start}
	node [block, right of=start] (ca) {Stadt A}
	node [block, right of=ca] (target) {Ziel}
	node [block, below of=ca] (cb) {Stadt B};
	\draw[->](start) -- node {} (ca);
	\draw[->](start) -- node {} (cb);
	\draw[->](cb) -- node {} (target);
\end{tikzpicture}
\end{center}

Der Agent würde hier nie zum Ziel kommen (s.
Kap.~\ref{sec:erreicht_der_agent_immer_sein_ziel?}), ein zufällig agierender Agent
hätte hier zumindest eine Wahrscheinlichkeit ungleich Null, irgendwann zum Ziel
zu kommen. Dies ist jedoch nicht garantiert, da auch hier bei ungünstigen
Entscheidungen der Agent in eine Endlosschleife gelangen könnte, ohne je ins
Ziel zu kommen.

% Aufgabe
\chapter{Aufgabe}
\label{chp:aufgabe2}

% - Zulässigkeit
\section{Zulässigkeit}
\label{sec:zulaessigkeit}

\begin{description}
    \item[$h_1(n)$: Luftlinienentfernung des zu $n$ gehörenden Knotens zum Startknoten] 
        Zulässig, da von jedem Knoten aus für einen Rundweg noch mindestens die
        Distanz zum Startknoten zurückgelegt werden muss. Die
        Luftlinienentfernung zum Startknoten ist dabei immer kleiner oder gleich
        der tatsächlich benötigten Distanz zum Startknoten (evtl.\ über mehrere
        Zwischenknoten).
    \item[$h_2(n)$: kürzester Pfad des zu n gehörenden Knotens zum Startknoten] 
        Zulässig, da von jedem Knoten aus für einen Rundweg noch mindestens die
        Distanz zum Startknoten zurückgelegt werden muss. Der kürzeste Pfad von
        diesem Knoten entspricht genau dieser Distanz.
    \item[$h_3(n)$: Summe der Entfernungen der im Zustand $n$ noch nicht
        besuchten Knoten] \textbf{zu dem Nachbarknoten mit der geringsten
        Entfernung}

        Zulässig, da die Strecke zu jedem noch unbesuchten Knoten noch
        zurückgelegt werden muss und diese im günstigsten Fall nur jene zum
        Nachbarknoten ist.
    \item[$h_4(n)$: Summe der Luftlinienentfernungen aller im Zustand $n$ noch
        nicht besuchten] \textbf{Knoten zum Startknoten}

        Nicht zulässig, Gegenbeispiel (noch keine Stadt besucht):

        \begin{center}
        \vspace{0.4cm}
        \begin{tikzpicture}[auto, thick, node distance=4cm, >=triangle 45]
        \draw
            node at (4, 0)[block] (start) {Start}
            node at (4, -8)[block] (cb) {Stadt B}
            node [block, left of=cb] (ca) {Stadt A}
            node [block, right of=cb] (cc) {Stadt C}
            node at (4, -10) [block] (cd) {Stadt D} ;
            \draw[-](start) -- node {$\sqrt{20}$} (ca);
            \draw[-](ca) -- node {2} (cb);
            \draw[-](cb) -- node {2} (cc);
            \draw[-](cb) -- node {1} (cd);
            \draw[-](start) -- node {$\sqrt{20}$} (cc);
        \end{tikzpicture}
        \end{center}

        \[ h_4(\text{Stadt A}) = 4 + 5 + \sqrt{20} \]
        \[ k(\text{Stadt A}) = 2 + 1 + 1 + 2 + \sqrt{20} \]
\end{description} 

% - Dominanz
\section{Dominanz}
\label{sec:dominanz}

\begin{itemize}
    \item $ h_1(n) \leq h_2(n) $ weil Luftlinienentfernung zu einem Knoten nie größer als
        Pfad zu einem Knoten (Dreiecksungleichung)

    \item $ h_3(n) $ kann nicht eingeordnet werden:
        \begin{itemize}
            \item $ h_3(n) < h_1(n) $ wenn n und die anderen nicht besuchten Knoten dicht
                zusammen liegen, aber weit vom Startknoten entfernt sind.
            \item $ h_3(n) > h_2(n) $ wenn n nahe am Startknoten, die anderen
                nicht besuchten Knoten aber weit von einander entfernt sind.
        \end{itemize}

    \item $ h_4(n) $ ist nicht zulässig $\rightarrow$ Voraussetzung für Dominanz
        nicht erfüllt.
\end{itemize}

% Aufgabe
\chapter{Aufgabe}
\label{chp:aufgabe3}

% - Variablen, Wertebereiche
\section{Variablen, Wertebereiche}
\label{sec:variablen_wertebereiche}

\begin{itemize}
    \item Variablen:
        \begin{itemize}
            \item $ \text{Feld}_{i,j} $ $\rightarrow$ Zahl im Feld in Zeile $ i
                $ und Spalte $ j $
            \item $ i, j \in \left\{ 0, 1, 2, 3 \right\} $
            \item $ \text{Feld}_{0,0} $ befindet sich in der oberen linken Ecke.
        \end{itemize}
    \item (Basis-)Wertebereiche:
        \begin{itemize}
            \item $ \text{Feld}_{i,j} \in \left\{ 1, 2, 3, 4 \right\}, \quad i, j \in \left\{ 0, 1, 2, 3 \right\} $
        \end{itemize}
    \item Constraints (nicht verlangt):
        \begin{itemize}
            \item $ \text{alldiff}(\text{Feld}_{0,j}, \text{Feld}_{1,j}, \text{Feld}_{2,j}, \text{Feld}_{3,j}), \quad j \in \left\{ 0, 1, 2, 3 \right\} $
            \item $ \text{alldiff}(\text{Feld}_{i,0}, \text{Feld}_{i,1}, \text{Feld}_{i,2}, \text{Feld}_{i,3}), \quad j \in \left\{ 0, 1, 2, 3 \right\} $
            \item $ \text{alldiff}(\text{Feld}_{m,n}, \text{Feld}_{m,n+1}, \text{Feld}_{m+1,n}, \text{Feld}_{m+1,n+1}), \quad m, n \in \left\{ 0, 2\right\} $
        \end{itemize}
\end{itemize}

% - aktuelle Wertebereiche der Variablen
\section{aktuelle Wertebereiche der Variablen}
\label{sec:aktuelle_wertebereiche_der_variablen}

\begin{itemize}
    \item $ \text{Feld}_{0,0} \in \left\{ 2, 3 \right\} $
    \item $ \text{Feld}_{0,1} \in \left\{ 2, 3 \right\} $
    \item $ \text{Feld}_{0,2} \in \left\{ 1, 4 \right\} $
    \item $ \text{Feld}_{0,3} \in \left\{ 3 \right\} $ (bereits zugewiesen)
    \item $ \text{Feld}_{1,0} \in \left\{ 2, 3, 4 \right\} $
    \item $ \text{Feld}_{1,1} \in \left\{ 1 \right\} $ (bereits zugewiesen)
    \item $ \text{Feld}_{1,2} \in \left\{ 4 \right\} $
    \item $ \text{Feld}_{1,3} \in \left\{ 2 \right\} $
    \item $ \text{Feld}_{2,0} \in \left\{ 1, 3 \right\} $
    \item $ \text{Feld}_{2,1} \in \left\{ 3 \right\} $
    \item $ \text{Feld}_{2,2} \in \left\{ 2 \right\} $ (bereits zugewiesen)
    \item $ \text{Feld}_{2,3} \in \left\{ 4 \right\} $ (bereits zugewiesen)
    \item $ \text{Feld}_{3,0} \in \left\{ 1, 2, 3, 4 \right\} $
    \item $ \text{Feld}_{3,1} \in \left\{ 2, 3, 4 \right\} $
    \item $ \text{Feld}_{3,2} \in \left\{ 1, 3 \right\} $
    \item $ \text{Feld}_{3,3} \in \left\{ 1 \right\} $
\end{itemize}

% - Lösen des Sudokus
\section{Lösen des Sudokus}
\label{sec:loesen_des_sudokus}

$\rightarrow$ Anwenden der Minimum-Remaining-Values-Heuristik (MRV):

1.\ Iteration:

\begin{itemize}
    \item Wählen der noch nicht zugewiesenen Variable mit dem kleinsten
        Wertebereich, z.B.\ $ \text{Feld}_{3,3} \in \left\{ 1 \right\} $.
    \item Probieren des ersten Wertes (hier der einzige Wert): $
        \text{Feld}_{3,3} = 1 $
    \item Forward-Checking ändert Wertebereiche:
        \begin{itemize}
            \item $ \text{Feld}_{3,2} \in \left\{ 3 \right\} $
            \item $ \text{Feld}_{3,0} \in \left\{ 2, 3, 4 \right\} $
        \end{itemize}
\end{itemize}

2.\ Iteration:

\begin{itemize}
    \item Wählen der noch nicht zugewiesenen Variable mit dem kleinsten
        Wertebereich, z.B.\ $ \text{Feld}_{3,2} \in \left\{ 3 \right\} $.
    \item Probieren des ersten Wertes (hier der einzige Wert): $
        \text{Feld}_{3,2} = 3 $
    \item Forward-Checking ändert Wertebereiche:
        \begin{itemize}
            \item $ \text{Feld}_{3,0} \in \left\{ 2, 4 \right\} $
            \item $ \text{Feld}_{3,1} \in \left\{ 2, 4 \right\} $
        \end{itemize}
\end{itemize}

3.\ Iteration:

\begin{itemize}
    \item Wählen der noch nicht zugewiesenen Variable mit dem kleinsten
        Wertebereich, z.B.\ $ \text{Feld}_{1,2} \in \left\{ 4 \right\} $.
    \item Probieren des ersten Wertes (hier der einzige Wert): $
        \text{Feld}_{1,2} = 4 $
    \item Forward-Checking ändert Wertebereiche:
        \begin{itemize}
            \item $ \text{Feld}_{0,2} \in \left\{ 1 \right\} $
            \item $ \text{Feld}_{1,0} \in \left\{ 2, 3 \right\} $
        \end{itemize}
\end{itemize}

4.\ Iteration:

\begin{itemize}
    \item Wählen der noch nicht zugewiesenen Variable mit dem kleinsten
        Wertebereich, z.B.\ $ \text{Feld}_{0,2} \in \left\{ 1 \right\} $
    \item Probieren des ersten Wertes (hier der einzige Wert): $
        \text{Feld}_{0,2} = 1 $
\end{itemize}

% Aufgabe
\chapter{Aufgabe}
\label{chp:aufgabe4}

% - Beschreibung des Problems
\section{Beschreibung des Problems}
\label{sec:beschreibung_des_problems}

$\rightarrow$ atomare Sätze sind entweder
\begin{itemize}
    \item z.B.\ \verb+Antilope_rot+ ist wahr, wenn die Antilope die rote Karte gezogen hat und
        \verb+Antilope_blau+ ist wahr, wenn die Antilope die blaue Karte gezogen
        hat oder
    \item leichter zu vereinfachen:
        
        Es existiert nur z.B.\ \verb+Antilope_rot+,
        denn \verb+Antilope_blau+ = $ \neg $\verb+Antilope_rot+
\end{itemize}

% - Sätze für Giraffe und Elefant
\section{Sätze für Giraffe und Elefant}
\label{sec:saetze_fuer_giraffe_und_elefant}

\begin{itemize}
    \item Antilope sagt: Papagei rot, Hyäne blau
        \begin{align*}
            (\texttt{Antilope\_rot} \Rightarrow (
                (\texttt{Papagei\_rot} \land \texttt{Hyäne\_blau}) \lor
                (\neg \texttt{Papagei\_rot} \land \neg \texttt{Hyäne\_blau})
                )
            ) \land \\
            (\texttt{Antilope\_blau} \Rightarrow (
                (\neg \texttt{Papagei\_rot} \land \texttt{Hyäne\_blau}) \lor
                (\texttt{Papagei\_rot} \land \neg \texttt{Hyäne\_blau})
                )
            )
        \end{align*}
        Mit $ \texttt{*\_blau} \Leftrightarrow \neg \texttt{*\_rot} $:
        \begin{align*}
            (\texttt{Antilope\_rot} \Rightarrow (
                (\texttt{Papagei\_rot} \land \neg \texttt{Hyäne\_rot}) \lor
                (\neg \texttt{Papagei\_rot} \land \texttt{Hyäne\_rot})
                )
            ) \land \\
            (\neg \texttt{Antilope\_rot} \Rightarrow (
                (\neg \texttt{Papagei\_rot} \land \neg\texttt{Hyäne\_rot}) \lor
                (\texttt{Papagei\_rot} \land \texttt{Hyäne\_rot})
                )
            )
        \end{align*}
    \item Elefant sagt: Papagei blau, Schlange rot
        \begin{align*}
            (\texttt{Elefant\_rot} \Rightarrow (
                (\texttt{Papagei\_blau} \land \texttt{Schlange\_rot}) \lor
                (\neg \texttt{Papagei\_blau} \land \neg \texttt{Schlange\_rot})
                )
            ) \land \\
            (\texttt{Elefant\_blau} \Rightarrow (
                (\neg \texttt{Papagei\_blau} \land \texttt{Schlange\_rot}) \lor
                (\texttt{Papagei\_blau} \land \neg \texttt{Schlange\_rot})
                )
            )
        \end{align*}
        Mit $ \texttt{*\_blau} \Leftrightarrow \neg \texttt{*\_rot} $:
        \begin{align*}
            (\texttt{Elefant\_rot} \Rightarrow (
                (\neg \texttt{Papagei\_rot} \land \texttt{Schlange\_rot}) \lor
                (\texttt{Papagei\_rot} \land \neg \texttt{Schlange\_rot})
                )
            ) \land \\
            (\neg \texttt{Elefant\_rot} \Rightarrow (
                (\texttt{Papagei\_rot} \land \texttt{Schlange\_rot}) \lor
                (\neg \texttt{Papagei\_rot} \land \neg \texttt{Schlange\_rot})
                )
            )
        \end{align*}
\end{itemize}

% - Resolution
\section{Resolution}
\label{sec:resolution}

Können Sie die Fragestellung an das Faultier durch einmalige Anwendung der
Resolution lösen? $\rightarrow$ Nein, da mit dem Algorithmus immer nur versucht
werden kann, ein einzelnes Literal oder dessen Negation aus den vorhandenen
Sätzen abzuleiten.

Zur Anwendung des Resolutionsalgorithmus wird die Wissensbasis zunächst in die
Konjunktive Normalform gebracht:

\begin{align*}
    \text{KB} =\, &
            (\texttt{Antilope\_rot} \Rightarrow (
                (\texttt{Papagei\_rot} \land \neg \texttt{Hyäne\_rot}) \lor
                (\neg \texttt{Papagei\_rot} \land \texttt{Hyäne\_rot})
                )
            ) \land \\ &
            (\neg \texttt{Antilope\_rot} \Rightarrow (
                (\neg \texttt{Papagei\_rot} \land \neg\texttt{Hyäne\_rot}) \lor
                (\texttt{Papagei\_rot} \land \texttt{Hyäne\_rot})
                )
            ) \land \\ &
            (\texttt{Elefant\_rot} \Rightarrow (
                (\neg \texttt{Papagei\_rot} \land \texttt{Schlange\_rot}) \lor
                (\texttt{Papagei\_rot} \land \neg \texttt{Schlange\_rot})
                )
            ) \land \\ &
            (\neg \texttt{Elefant\_rot} \Rightarrow (
                (\texttt{Papagei\_rot} \land \texttt{Schlange\_rot}) \lor
                (\neg \texttt{Papagei\_rot} \land \neg \texttt{Schlange\_rot})
                )
            ) \\
        =\, & 
            (\neg \texttt{Antilope\_rot} \lor (
                (\texttt{Papagei\_rot} \land \neg \texttt{Hyäne\_rot}) \lor
                (\neg \texttt{Papagei\_rot} \land \texttt{Hyäne\_rot})
                )
            ) \land \\ &
            (\texttt{Antilope\_rot} \lor (
                (\neg \texttt{Papagei\_rot} \land \neg\texttt{Hyäne\_rot}) \lor
                (\texttt{Papagei\_rot} \land \texttt{Hyäne\_rot})
                )
            ) \land \\ &
            (\neg \texttt{Elefant\_rot} \lor (
                (\neg \texttt{Papagei\_rot} \land \texttt{Schlange\_rot}) \lor
                (\texttt{Papagei\_rot} \land \neg \texttt{Schlange\_rot})
                )
            ) \land \\ &
            (\texttt{Elefant\_rot} \lor (
                (\texttt{Papagei\_rot} \land \texttt{Schlange\_rot}) \lor
                (\neg \texttt{Papagei\_rot} \land \neg \texttt{Schlange\_rot})
                )
            ) \\
        =\, & 
            (\neg \texttt{Antilope\_rot} \lor ( \\ & \quad
                ((\texttt{Papagei\_rot} \land \neg \texttt{Hyäne\_rot}) \lor
                    \neg \texttt{Papagei\_rot}) \land \\ & \quad
                ((\texttt{Papagei\_rot} \land \neg \texttt{Hyäne\_rot}) \lor
                    \texttt{Hyäne\_rot}) \\ & \quad
                )
            ) \land \\ &
            (\texttt{Antilope\_rot} \lor ( \\ & \quad
                ((\neg \texttt{Papagei\_rot} \land \neg\texttt{Hyäne\_rot}) \lor
                    \texttt{Papagei\_rot}) \land \\ & \quad
                ((\neg \texttt{Papagei\_rot} \land \neg\texttt{Hyäne\_rot}) \lor
                    \texttt{Hyäne\_rot}) \\ & \quad
                )
            ) \land \\ &
            (\neg \texttt{Elefant\_rot} \lor ( \\ & \quad
                ((\neg \texttt{Papagei\_rot} \land \texttt{Schlange\_rot}) \lor
                    \texttt{Papagei\_rot}) \land \\ & \quad
                ((\neg \texttt{Papagei\_rot} \land \texttt{Schlange\_rot}) \lor
                    \neg \texttt{Schlange\_rot}) \\ & \quad
                )
            ) \land \\ &
            (\texttt{Elefant\_rot} \lor ( \\ & \quad
                ((\texttt{Papagei\_rot} \land \texttt{Schlange\_rot}) \lor
                    \neg \texttt{Papagei\_rot}) \land \\ & \quad
                ((\texttt{Papagei\_rot} \land \texttt{Schlange\_rot}) \lor
                    \neg \texttt{Schlange\_rot}) \\ & \quad
                )
            ) \\
        =\, & 
            (\neg \texttt{Antilope\_rot} \lor ( \\ & \quad
                ((\underbrace{\texttt{Papagei\_rot} \lor \neg \texttt{Papagei\_rot}}_{\text{True}}) \land
                (\neg \texttt{Hyäne\_rot} \lor \neg \texttt{Papagei\_rot})) \land \\ & \quad
                ((\texttt{Papagei\_rot} \lor \texttt{Hyäne\_rot}) \land
                (\underbrace{\neg \texttt{Hyäne\_rot} \lor \texttt{Hyäne\_rot}}_{\text{True}})) \land \\ & \quad
            ) \land \\ &
            (\texttt{Antilope\_rot} \lor ( \\ & \quad
                ((\neg \texttt{Papagei\_rot} \lor \texttt{Papagei\_rot}) \land
                (\neg\texttt{Hyäne\_rot}) \lor \texttt{Papagei\_rot}) \land \\ & \quad
                ((\neg \texttt{Papagei\_rot} \lor \texttt{Hyäne\_rot}) \land
                (\neg\texttt{Hyäne\_rot} \lor \texttt{Hyäne\_rot})) \\ & \quad
            ) \land \\ &
            (\neg \texttt{Elefant\_rot} \lor ( \\ & \quad
                ((\neg \texttt{Papagei\_rot} \lor \texttt{Papagei\_rot}) \land
                (\texttt{Schlange\_rot}) \lor \texttt{Papagei\_rot}) \land \\ & \quad
                ((\neg \texttt{Papagei\_rot} \lor \neg \texttt{Schlange\_rot}) \land
                (\texttt{Schlange\_rot} \lor \neg \texttt{Schlange\_rot})) \\ & \quad
            ) \land \\ &
            (\texttt{Elefant\_rot} \lor ( \\ & \quad
                ((\texttt{Papagei\_rot} \lor \neg \texttt{Papagei\_rot}) \land
                (\texttt{Schlange\_rot} \lor \neg \texttt{Papagei\_rot})) \land \\ & \quad
                ((\texttt{Papagei\_rot} \lor \neg \texttt{Schlange\_rot}) \land
                (\texttt{Schlange\_rot} \lor \neg \texttt{Schlange\_rot})) \\ & \quad
            ) \\
\end{align*}

\begin{align*}
        =\, & 
            (\neg \texttt{Antilope\_rot} \lor ( 
                (\neg \texttt{Hyäne\_rot} \lor \neg \texttt{Papagei\_rot}) \land
                (\texttt{Papagei\_rot} \lor \texttt{Hyäne\_rot})
            )) \land \\ &
            (\texttt{Antilope\_rot} \lor ( 
                (\neg\texttt{Hyäne\_rot} \lor \texttt{Papagei\_rot}) \land
                (\neg \texttt{Papagei\_rot} \lor \texttt{Hyäne\_rot})
            )) \land \\ &
            (\neg \texttt{Elefant\_rot} \lor (
                (\texttt{Schlange\_rot} \lor \texttt{Papagei\_rot}) \land
                (\neg \texttt{Papagei\_rot} \lor \neg \texttt{Schlange\_rot})
            )) \land \\ &
            (\texttt{Elefant\_rot} \lor ( 
                (\texttt{Schlange\_rot} \lor \neg \texttt{Papagei\_rot}) \land
                (\texttt{Papagei\_rot} \lor \neg \texttt{Schlange\_rot})
            )) \\
        =\, & 
            (\neg \texttt{Antilope\_rot} \lor \neg \texttt{Hyäne\_rot} \lor \neg \texttt{Papagei\_rot}) \land \tag{A} \\ &
            (\neg \texttt{Antilope\_rot} \lor \texttt{Papagei\_rot} \lor \texttt{Hyäne\_rot}) \land \tag{B} \\ &
            (\texttt{Antilope\_rot} \lor \neg\texttt{Hyäne\_rot} \lor \texttt{Papagei\_rot}) \land \tag{C} \\ &
            (\texttt{Antilope\_rot} \lor \neg \texttt{Papagei\_rot} \lor \texttt{Hyäne\_rot}) \land \tag{D} \\ &
            (\neg \texttt{Elefant\_rot} \lor \texttt{Schlange\_rot} \lor \texttt{Papagei\_rot}) \land \tag{E} \\ &
            (\neg \texttt{Elefant\_rot} \lor \neg \texttt{Papagei\_rot} \lor \neg \texttt{Schlange\_rot}) \land \tag{F} \\ &
            (\texttt{Elefant\_rot} \lor \texttt{Schlange\_rot} \lor \neg \texttt{Papagei\_rot}) \land \tag{G} \\ &
            (\texttt{Elefant\_rot} \lor \texttt{Papagei\_rot} \lor \neg \texttt{Schlange\_rot}) \tag{H}
\end{align*}

Resolutionsalgorithmus: Um zu zeigen, dass $ \texttt{Elefant\_rot} $ gilt, nehmen
wir $ \neg \texttt{Elefant\_rot} $ (I) an. Dann leiten wir damit neue Sätze ab:

\begin{itemize}
    \item Aus Kombination von I und E folgt: $ \texttt{Schlange\_rot} \lor \texttt{Papagei\_rot} $ (J)
    \item Aus Kombination von J und G folgt: $ \texttt{Elefant\_rot} \lor \texttt{Schlange\_rot} $ (K)
    \item Aus Kombination von K und H folgt: $ \texttt{Elefant\_rot} \lor \texttt{Papagei\_rot} $ (L)
    \item Aus Kombination von L und I folgt: $ \texttt{Papagei\_rot} $ (M)
    \item Aus Kombination von M und D folgt: $ \texttt{Antilope\_rot} \lor \texttt{Hyäne\_rot} $ (N)
\end{itemize}

% End of Text

\cleardoublepage{}
\renewcommand*{\chapterpagestyle}{scrheadings}
\renewcommand*{\indexpagestyle}{scrheadings}
\pagenumbering{Roman}

\end{document}
